\documentclass{beamer}
\usetheme{Boadilla}

\makeatother
\setbeamertemplate{footline}
{
    \leavevmode%
    \hbox{%
    \begin{beamercolorbox}[wd=.4\paperwidth,ht=2.25ex,dp=1ex,center]{author in head/foot}%
        \usebeamerfont{author in head/foot}\insertshortauthor
    \end{beamercolorbox}%
    \begin{beamercolorbox}[wd=.55\paperwidth,ht=2.25ex,dp=1ex,center]{title in head/foot}%
        \usebeamerfont{title in head/foot}\insertshorttitle
    \end{beamercolorbox}%
    \begin{beamercolorbox}[wd=.05\paperwidth,ht=2.25ex,dp=1ex,center]{date in head/foot}%
        \insertframenumber{}
    \end{beamercolorbox}}%
    \vskip0pt%
}
\makeatletter
\setbeamertemplate{navigation symbols}{}

\usepackage[T1]{fontenc}
\usepackage{lmodern}
\usepackage{amssymb,amsmath}
\renewcommand{\familydefault}{\sfdefault}

\DeclareMathOperator*{\argmax}{argmax}

\usepackage{mathtools}
\usepackage{graphicx}
\usepackage{threeparttable}
\usepackage{booktabs}
\usepackage{siunitx}
\sisetup{parse-numbers=false}

%\setlength{\OuterFrameSep}{-2pt}
%\makeatletter
%\preto{\@verbatim}{\topsep=-10pt \partopsep=-10pt }
%\makeatother

\title[Lecture 20:\ Individual-Specific Parameters II]{Lecture 20:\ Individual-Specific Parameters II}
\author[ResEcon 703:\ Advanced Econometrics]{ResEcon 703:\ Topics in Advanced Econometrics}
\date{Matt Woerman\\University of Massachusetts Amherst}

\begin{document}

{\setbeamertemplate{footline}{} 
\begin{frame}[noframenumbering]
    \titlepage
\end{frame}
}

\begin{frame}\frametitle{Agenda}
    Last time
    \begin{itemize}
        \item Individual-Specific Parameters
    \end{itemize}
    \vspace{2ex}
    Today
    \begin{itemize}
        \item Individual-Specific Parameters Example in R
    \end{itemize}
    \vspace{2ex}
    Upcoming
    \begin{itemize}
        \item Reading for next time
        \begin{itemize}
            \item Train textbook, Chapter 7.7
        \end{itemize}
        \item Problem sets
        \begin{itemize}
            \item Problem Set 4 is posted, due November 21
        \end{itemize}
    \end{itemize}
\end{frame}

\begin{frame}\frametitle{Individual-Specific Parameters}
    We can combine the unconditional (or population) distribution of coefficients, $f(\beta \mid \theta)$, and an individual's choices, $y_n$, and data, $x_n$, to generate the conditional distribution of coefficients, $h(\beta \mid y_n, x_n, \theta)$
    $$h(\beta \mid y_n, x_n, \theta) = \frac{P(y_n \mid x_n, \beta) \times f(\beta \mid \theta)}{P(y_n \mid x_n, \theta)}$$ \\
    \vspace{2ex}
    If we only want the mean of this distribution, we can simulate the conditional mean coefficients as
    $$\check{\beta}_n = \frac{\sum_{r = 1}^R \beta^r P(y_n \mid x_n, \beta^r)}{\sum_{r = 1}^R P(y_n \mid x_n, \beta^r)}$$ \\
    \vspace{2ex}
    With enough observed choices for an individual ($T > 50$?), these means converge to the individual's coefficients
\end{frame}

\begin{frame}\frametitle{}
    \vfill
    \centering
    \begin{beamercolorbox}[center]{title}
        \Large Simulation-Based Estimation Example in R
    \end{beamercolorbox}
    \vfill
\end{frame}

\begin{frame}\frametitle{Individual-Specific Parameters Example}
    We are again studying how consumers make choices about expensive and highly energy-consuming systems in their homes. We have data on 250 households in California and the type of HVAC (heating, ventilation, and air conditioning) system in their home. Each household has the following choice set, and we observe the following data \\
    \vspace{3ex}
    \begin{columns}
    	\begin{column}{0.5\textwidth}
		    Choice set
		    \begin{itemize}
		    	\item GCC: gas central with AC
		    	\item ECC: electric central with AC
		    	\item ERC: electric room with AC
		    	\item HPC: heat pump with AC
		    	\item GC: gas central
		    	\item EC: electric central
		    	\item ER: electric room
		    \end{itemize}
		    \vspace{2ex}
	    \end{column}
	    \begin{column}{0.5\textwidth}
		    Alternative-specific data
		    \begin{itemize}
		    	\item ICH: installation cost for heat
		    	\item ICCA: installation cost for AC
		    	\item OCH: operating cost for heat
		    	\item OCCA: operating cost for AC
		    \end{itemize}
		    \vspace{2ex}
		    Household demographic data
		    \begin{itemize}
		    	\item income: annual income
		    \end{itemize}
		\end{column}
    \end{columns}
\end{frame}

\begin{frame}[fragile]\frametitle{Load Dataset}
    <<R CODE HERE>>
\end{frame}

\begin{frame}[fragile]\frametitle{Dataset}
    <<R CODE HERE>>
\end{frame}

\begin{frame}[fragile]\frametitle{Format Dataset in a Long Format}
    <<R CODE HERE>>
\end{frame}

\begin{frame}[fragile]\frametitle{Dataset in a Long Format}
    <<R CODE HERE>>
\end{frame}

\begin{frame}[fragile]\frametitle{Clean Dataset}
    <<R CODE HERE>>
\end{frame}

\begin{frame}[fragile]\frametitle{Cleaned Dataset}
    <<R CODE HERE>>
\end{frame}

\begin{frame}[fragile]\frametitle{Convert Dataset to \texttt{mlogit} Format}
    <<R CODE HERE>>
\end{frame}

\begin{frame}[fragile]\frametitle{Dataset in \texttt{mlogit} Format}
    <<R CODE HERE>>
\end{frame}

\begin{frame}\frametitle{Conditional Mean Coefficients to Simulate}
    The representative utility of each alternative is
    $$V_{nj} = \alpha AC_j + \beta_1 IC_{nj} + \beta_2 OC_{nj}$$
    with
    $$\ln \alpha \sim N(\mu, \sigma^2)$$
    and $\beta_1$ and $\beta_2$ fixed (not random) \\
    \vspace{3ex}
    For each alternative, what is the mean $\alpha$ coefficient for the households with that HVAC system?
    \begin{enumerate}
        \item Estimate the mixed logit model
        \item Simulate $\check{\alpha}_n$ for each household
        \item Average $\check{\alpha}_n$ for the households with each HVAC system
    \end{enumerate}
\end{frame}

\begin{frame}\frametitle{Simulating Conditional Mean Coefficients}
    Two ways to simulate conditional mean coefficients for each household
    \begin{itemize}
        \item \texttt{mlogit} package
        \item Code the simulation by hand
    \end{itemize}
    \vspace{2ex}
    The \texttt{mlogit} function \texttt{fitted(model, type = `parameters')} simulates the conditional mean coefficients for every individual
    \begin{itemize}
        \item This function returns an $N \times K$ matrix of conditional mean coefficients
    \end{itemize}
    \vspace{2ex}
    We can instead code the simulation by hand
    \begin{itemize}
        \item We may want to simulate additional objects that are not part of the \texttt{mlogit} functionality
    \end{itemize}
\end{frame}

\begin{frame}[fragile]\frametitle{Mixed Logit Model Using \texttt{mlogit}}
    <<R CODE HERE>>
\end{frame}

\begin{frame}[fragile]\frametitle{Model Results Using \texttt{mlogit}}
    <<R CODE HERE>>
\end{frame}

\begin{frame}[fragile]\frametitle{Conditional Mean Coefficients Using \texttt{mlogit}}
    <<R CODE HERE>>
\end{frame}

\begin{frame}[fragile]\frametitle{Coefficients for Each Alternative Using \texttt{mlogit}}
    <<R CODE HERE>>
\end{frame}

\begin{frame}\frametitle{Steps for Simulating Conditional Mean Coefficients}
    $$\check{\beta}_n = \frac{\sum_{r = 1}^R \beta^r P(y_n \mid x_n, \beta^r)}{\sum_{r = 1}^R P(y_n \mid x_n, \beta^r)}$$
    \begin{enumerate}
        \item Draw $K \times N \times R$ standard normal random variables
        \begin{itemize}
            \item $K$ random coefficients for each of
            \item $N$ different decision makers for each of
            \item $R$ different simulation draws
        \end{itemize}
        \item Find the MSLE parameters, $\hat{\theta}$
        \begin{itemize}
            \item See slides from last week on MSLE
        \end{itemize}
        \item Simulate conditional mean coefficients using the MSLE parameters, $\hat{\theta}$
        \begin{enumerate}
            \item Transform each set of $K$ standard normals using $\hat{\theta}$ to get a set of $\beta_n^r$
            \item Calculate the choice probability of the chosen alternative for each individual and draw, $P(y_n \mid x_n, \beta_n^r)$
            \item Take a weighted average of $\beta_n^r$, with weights equal to $P(y_n \mid x_n, \beta_n^r)$, to get $\check{\beta}_n$ for each individual
        \end{enumerate}
    \end{enumerate}
\end{frame}

\begin{frame}[fragile]\frametitle{Step 1: Draw Random Variables (and Organize Data)}
    <<R CODE HERE>>
\end{frame}

\begin{frame}[fragile]\frametitle{Step 2a: Simulate Choice Probabilities for One Household}
    <<R CODE HERE>>
\end{frame}

\begin{frame}[fragile]\frametitle{Step 2b: Calculate Simulated Log-Likelihood}
    <<R CODE HERE>>
\end{frame}

\begin{frame}[fragile]\frametitle{Step 2c: Maximize Simulated Log-Likelihood}
    <<R CODE HERE>>
\end{frame}

\begin{frame}[fragile]\frametitle{Step 2d: Report MSLE Results}
    <<R CODE HERE>>
\end{frame}

\begin{frame}[fragile]\frametitle{Step 3a: Simulate Coefficients for One Household}
    $$\check{\beta}_n = \frac{\sum_{r = 1}^R \beta^r P(y_n \mid x_n, \beta^r)}{\sum_{r = 1}^R P(y_n \mid x_n, \beta^r)}$$
    <<R CODE HERE>>
\end{frame}

\begin{frame}[fragile]\frametitle{Step 3a: Simulate Coefficients for One Household}
    $$\check{\beta}_n = \frac{\sum_{r = 1}^R \beta^r P(y_n \mid x_n, \beta^r)}{\sum_{r = 1}^R P(y_n \mid x_n, \beta^r)}$$
    <<R CODE HERE>>
\end{frame}

\begin{frame}[fragile]\frametitle{Step 3b: Simulate Coefficients for All Household}
    $$\check{\beta}_n = \frac{\sum_{r = 1}^R \beta^r P(y_n \mid x_n, \beta^r)}{\sum_{r = 1}^R P(y_n \mid x_n, \beta^r)}$$
    <<R CODE HERE>>
\end{frame}

\begin{frame}[fragile]\frametitle{Conditional Mean Coefficients for Each HVAC System}
    <<R CODE HERE>>
\end{frame}

\begin{frame}\frametitle{Announcements}
    Reading for next time
    \begin{itemize}
        \item Train textbook, Chapter 7.7
    \end{itemize}
    \vspace{3ex}
    Upcoming
    \begin{itemize}
        \item Problem Set 4 is posted, due November 21
    \end{itemize}
\end{frame}

\end{document}