\documentclass{beamer}
\usetheme{Boadilla}

\makeatother
\setbeamertemplate{footline}
{
    \leavevmode%
    \hbox{%
    \begin{beamercolorbox}[wd=.4\paperwidth,ht=2.25ex,dp=1ex,center]{author in head/foot}%
        \usebeamerfont{author in head/foot}\insertshortauthor
    \end{beamercolorbox}%
    \begin{beamercolorbox}[wd=.55\paperwidth,ht=2.25ex,dp=1ex,center]{title in head/foot}%
        \usebeamerfont{title in head/foot}\insertshorttitle
    \end{beamercolorbox}%
    \begin{beamercolorbox}[wd=.05\paperwidth,ht=2.25ex,dp=1ex,center]{date in head/foot}%
        \insertframenumber{}
    \end{beamercolorbox}}%
    \vskip0pt%
}
\makeatletter
\setbeamertemplate{navigation symbols}{}

\usepackage[T1]{fontenc}
\usepackage{lmodern}
\usepackage{amssymb,amsmath}
\renewcommand{\familydefault}{\sfdefault}

\DeclareMathOperator*{\argmax}{argmax}

\usepackage{mathtools}
\usepackage{graphicx}
\usepackage{threeparttable}
\usepackage{booktabs}
\usepackage{siunitx}
\sisetup{parse-numbers=false}

%\setlength{\OuterFrameSep}{-2pt}
%\makeatletter
%\preto{\@verbatim}{\topsep=-10pt \partopsep=-10pt }
%\makeatother

\title[Lecture 24:\ Endogeneity II]{Lecture 24:\ Endogeneity II}
\author[ResEcon 703:\ Advanced Econometrics]{ResEcon 703:\ Topics in Advanced Econometrics}
\date{Matt Woerman\\University of Massachusetts Amherst}

\begin{document}

{\setbeamertemplate{footline}{} 
\begin{frame}[noframenumbering]
    \titlepage
\end{frame}
}

\begin{frame}\frametitle{Agenda}
    Last time
    \begin{itemize}
        \item Endogeneity
    \end{itemize}
    \vspace{2ex}
    Today
    \begin{itemize}
        \item Berry, Levinsohn, and Pakes (1995)
    \end{itemize}
    \vspace{2ex}
    Upcoming
    \begin{itemize}
        \item Final exam
        \begin{itemize}
        	\item Final exam is posted, due December 19
        \end{itemize}
        \item Course surveys
        \begin{itemize}
            \item Course surveys are open, due December 22
            \item \href{http://owl.umass.edu/partners/courseEvalSurvey/uma/}{Click here (\texttt{owl.umass.edu/partners/courseEvalSurvey/uma})}
        \end{itemize}
    \end{itemize}
\end{frame}

\begin{frame}\frametitle{Endogeneity}
    In many empirical settings, our variable of interest is endogenous
    \begin{itemize}
    	\item The variable of interest is correlated with the error term
        \item The coefficient on this variable will be biased and cannot be interpreted as causal
        \begin{itemize}
            \item Sometimes the direction of the bias can be predicted, in which case the estimated coefficient can be interpreted as a bound, but not always
        \end{itemize}
    \end{itemize}
    \vspace{2ex}
    Example: price and unobserved quality
    \begin{itemize}
    	\item Products with better unobserved quality are likely to have a higher price, so price is an endogenous attribute of the product
    	\item If we try to estimate a coefficient on price (marginal utility of price or income), our estimate will be biased downward and may even have the wrong sign
    \end{itemize}
    \vspace{2ex}
    Correcting for this endogeneity in a nonlinear structural model is more difficult than in a linear model
\end{frame}

\begin{frame}\frametitle{}
    \vfill
    \centering
    \begin{beamercolorbox}[center]{title}
        \Large Berry, Levinsohn, and Pakes (1995)
    \end{beamercolorbox}
    \vfill
\end{frame}

\begin{frame}\frametitle{Automobile Prices in Market Equilibrium}
    This paper develops a structural model of demand and supply in the automobile market
    \begin{itemize}
    	\item Random coefficient model of demand
        \item Nash-Bertrand model of firm pricing
    \end{itemize}
    \vspace{2ex}
    Why do we care about automobiles?
    \begin{itemize}
        \item They are important in their own right
        \begin{itemize}
            \item An automobile is one of the largest purchases a typical household makes
            \item Automobiles are an important part of many policy debates, ranging from international trade to environmental regulation 
        \end{itemize}
        \item They are an example of an important kind of industry
        \begin{itemize}
            \item Differentiated product market with oligopolistic competition
            \item This estimation technique can be used to estimate elasticities and cost parameters in other industries
        \end{itemize}
    \end{itemize}
\end{frame}

\begin{frame}\frametitle{BLP Innovation}
    What is so innovative about the BLP approach?
    \begin{itemize}
        \item BLP estimate flexible substitution patterns (elasticities) from market-level data using a computationally efficient algorithm that corrects for price endogeneity
    \end{itemize}
    \vspace{1ex}
    Flexible substitution patterns
    \begin{itemize}
        \item BLP model distributions of consumer preferences over attributes
        \item Alternative: estimate $J (J - 1) / 2$ elasticity parameters
    \end{itemize}
    \vspace{1ex}
    Market-level data
    \begin{itemize}
        \item Market-level data are more readily available than consumer-level data
        \item Consumer-level data may not have sufficient power for all elasticities
    \end{itemize}
    \vspace{1ex}
    Computationally efficient algorithm that corrects for price endogeneity
    \begin{itemize}
        \item BLP solve a nonlinear simultaneous equations problem
        \item BLP develop a ``contraction mapping'' for efficient estimation
    \end{itemize}
\end{frame}

\begin{frame}\frametitle{Utility Model}
    There are $T$ markets ($t = 1, \ldots, T$), each with $J_t$ products ($j = 1, \ldots, J_t$)
    \begin{itemize}
        \item The ``outside good'' is indexed as $j = 0$
    \end{itemize}
    \vspace{2ex}
    The utility that consumer $i$ in market $t$ obtains from product $j$ is
    $$U(x_{jt}, \xi_{jt}, p_{jt}, \tau_i; \theta_1, \theta_2)$$ \\
    \vspace{-1ex}
    \begin{itemize}
        \item $x_{jt}$: vector of non-price attributes of product $j$ in market $t$
        \item $\xi_{jt}$: utility of unobserved attributes of product $j$ in market $t$
        \item $p_{jt}$: price of product $j$ in market $t$
        \item $\tau_i$: characteristics of individual $i$
        \item $\theta_1$, $\theta_2$: vectors of unknown parameters
    \end{itemize}
\end{frame}

\begin{frame}\frametitle{Utility Functional Form}
    A quasi linear utility function gives the utility expression
    \begin{align*}
    u_{ijt} &= \alpha_i (y_i - p_{jt}) + x_{jt} \beta_i + \xi_{jt} + \varepsilon_{ijt} \\
    \intertext{BLP use a slightly different expression that comes from a Cobb-Douglas utility function}
    u_{ijt} &= \alpha_i \ln(y_i - p_{jt}) + x_{jt} \beta_i + \xi_{jt} + \varepsilon_{ijt}
    \end{align*}
    but the first expression is easier to follow \\
    \vspace{2ex}
    Depending on the data and context, we could capture some components of $\xi_{jt}$ through brand and market dummy variables
    $$\xi_{jt} = \xi_j + \xi_t + \Delta \xi_{jt}$$\\
    \begin{itemize}
        \item See Nevo (2000) for details on estimation with these dummy variables
    \end{itemize}
\end{frame}

\begin{frame}\frametitle{Individual Characteristics and Coefficients}
    There are two types of individual characteristics
    \begin{itemize}
        \item $D_i \sim \hat{P}_D^*(D)$: ``observed'' demographic characteristics
        \item $\nu_i \sim P_{\nu}^*(\nu)$: additional unobserved characteristics
    \end{itemize}
    BLP actually do not observe any individual-specific characteristics, but they do observe the distribution of demographic characteristics in the population, whereas $\nu_i$ represents characteristics that are fully unobserved \\
    \vspace{2ex}
    Individual coefficients are a function of these individual characteristics
    $$\begin{pmatrix}
        \alpha_i \\
        \beta_i
    \end{pmatrix} = \begin{pmatrix}
        \alpha \\
        \beta
    \end{pmatrix} + \Pi D_i + \Sigma \nu_i$$ \\
    \vspace{-1ex}
    \begin{itemize}
        \item $\Pi$ measures how individual coefficients vary with demographics
        \item $\Sigma$ represents variation in individual coefficients for unobserved reasons
    \end{itemize}
    \vspace{2ex}
    BLP have data on the income distribution, but no other demographics
\end{frame}

\begin{frame}\frametitle{Outside Good}
    A consumer may choose not to purchase any good, which is denoted the ``outside good'' and indexed as $j = 0$ \\
    \vspace{2ex}
    The utility that consumer $i$ in market $t$ obtains from the outside good is
    $$u_{i0t} = \alpha_i y_i + \xi_{0t} + \pi_0 D_i + \sigma_0 \nu_{i0} + \varepsilon_{i0t}$$ \\
    \vspace{2ex}
    The terms $\xi_{0t}$, $\pi_0$, and $\sigma_0$ are set equal to zero
    \begin{itemize}
        \item These terms are not identified without normalizing one $\xi_{jt}$ and components of $\Pi$ and $\Sigma$
    \end{itemize}
    \vspace{2ex}
    Then the utility of the outside good is normalized to be zero (in expectation)
    \begin{itemize}
        \item The utility of income, $\alpha_i y_i$, is common to all products, so it effectively disappears
    \end{itemize}
\end{frame}

\begin{frame}\frametitle{Full Demand Model}
    The utility that consumer $i$ in market $t$ obtains from product $j$ is
    \begin{align*}
        &u_{ijt} = \alpha_i y_i + \delta_{jt}(x_{jt}, p_{jt}, \xi_{jt}; \theta_1) + \mu_{ijt}(x_{jt}, p_{jt}, \nu_i, D_i; \theta_2) + \varepsilon_{ijt} \\
        \intertext{where}
        &\delta_{jt}(x_{jt}, p_{jt}, \xi_{jt}; \theta_1) = x_{jt} \beta - \alpha p_{jt} + \xi_{jt} \\
        &\mu_{ijt}(x_{jt}, p_{jt}, \nu_i, D_i; \theta_2) = (-p_{jt}, x_{jt}) (\Pi D_i + \Sigma \nu_i)
    \end{align*} \\
    \vspace{-1ex}
    \begin{itemize}
        \item $\alpha_i y_i$: common to all products and effectively disappears
        \item $\delta_{jt}$: mean utility of product $j$ in market $t$ that is common to all consumers
        \item $\mu_{ijt} + \varepsilon_{ijt}$: mean-zero deviation from the common utility that captures the random coefficients and idiosyncratic utility
    \end{itemize}
\end{frame}

\begin{frame}\frametitle{Market Shares}
    Consumer $i$ in market $t$ purchases product $j$ if and only if
    $$u_{ijt} \geq u_{i \ell t} \; \forall \ell \in J_t$$ \\
    \vspace{2ex}
    Conditional on all product attributes and model parameters, the set of individual characteristics that lead to the choice of product $j$ in market $t$ is
    $$A_{jt}(x_t, p_t, \delta_t; \theta_2) = \left\{ (D_i, \nu_i, \varepsilon_{i0t}, \ldots, \varepsilon_{iJ_tt}) \mid u_{ijt} \geq u_{i \ell t} \; \forall \ell \in J_t \right\}$$ \\
    \vspace{2ex}
    Then the market share of product $j$ in market $t$ is
    \begin{align*}
        s_{jt}(x_t, p_t, \delta_t; \theta_2) &= \int_{A_{jt}} dP^*(D, \nu, \varepsilon) \\
        &= \int_{A_{jt}} dP_{\varepsilon}^*(\varepsilon) dP_{\nu}^*(\nu) d\hat{P}_D^*(D)
    \end{align*}
    if $\varepsilon$, $\nu$, and $D$ are assumed to be independent
\end{frame}

\begin{frame}\frametitle{Endogeneity of Price}
    Product prices are not randomly assigned but set to maximize firm profits
    \begin{itemize}
        \item If firms know how consumers value the unobserved product attributes, $\xi_{jt}$, then they will consider these values when setting prices
        \item But then prices are correlated with these unobserved product attributes, so estimates of the price coefficient will be biased
        \item This problem is analogous to the classic simultaneity problem of demand and supply
    \end{itemize}
    \vspace{2ex}
    BLP propose instrumenting for price
    \begin{itemize}
        \item But they cannot use instruments in a nonlinear model
        \item BLP use an estimation method that transforms product attributes into a linear function of $\xi_{jt}$
    \end{itemize}
\end{frame}

\begin{frame}\frametitle{Cost Function}
    BLP do not just discuss how profit-maximizing firms lead to price endogeneity, they also explicitly model supply and the pricing decision of firms \\
    \vspace{2ex}
    There are $F$ firms ($f = 1, \ldots, F$), each of which produces subset $\mathcal{J}_f$ of the products \\
    \vspace{2ex}
    The marginal cost of producing product $j$ for market $t$, $mc_{jt}$, is given by
    $$\ln(mc_{jt}) = w_{jt} \gamma + \omega_{jt}$$ \\
    \vspace{-1ex}
    \begin{itemize}
        \item $w_{jt}$: vector of cost characteristics for product $j$ in market $t$, which can overlap with product attributes, $x_{jt}$
        \item $\gamma$: unknown parameters
        \item $\omega_{jt}$: unobserved component of marginal costs, which can be correlated with $\xi_{jt}$
    \end{itemize}
\end{frame}

\begin{frame}\frametitle{Profit-Maximizing Firms}
    The profits of firm $f$ in market $t$ is given by
    $$Profits_{ft} = \sum_{j \in \mathcal{J}_f} (p_{jt} - mc_{jt}) M_t s_{jt}(x_t, p_t, \delta_t; \theta_2)$$
    where $M_t$ is the number of consumers in market $t$ \\
    \vspace{2ex}
    BLP assume a Nash-Bertrand model of pricing, which yields the first-order condition for the price of product $j$ in market $t$
    \begin{itemize}
        \item Each firm sets prices to maximize its profits conditional on the attributes of all products and the prices of all other firms
    \end{itemize}
    \vspace{1ex}
    $$s_{jt}(x_t, p_t, \delta_t; \theta_2) + \sum_{\ell \in \mathcal{J}_f} (p_{\ell t} - mc_{\ell t}) \frac{\partial s_{\ell t}(x_t, p_t, \delta_t; \theta_2)}{\partial p_{jt}} = 0$$
\end{frame}

\begin{frame}\frametitle{Markups}
    The first-order condition can be expressed in vector notation as
    $$s_t(x_t, p_t, \delta_t; \theta_2) - \Delta_t(x_t, p_t, \delta_t; \theta_2)(p_t - mc_t) = 0$$
    where
    $$\Delta_{j \ell} = \begin{cases}
        \frac{- \partial s_{\ell t}}{\partial p_{jt}} & \text{if $j$ and $\ell$ are produced by the same firm} \\
        0 & \text{otherwise}
    \end{cases}$$ \\
    \vspace{2ex}
    Solving this first-order condition for prices gives an optimal markup rule
    \begin{align*}
        p_t &= mc + b_t(x_t, p_t, \delta_t; \theta_2) \\
        \intertext{where $b_t$ is the vector of markups in market $t$ given by}
        b_t(x_t, p_t, \delta_t; \theta_2) &= \Delta_t(x_t, p_t, \delta_t; \theta_2)^{-1} s_t(x_t, p_t, \delta_t; \theta_2)
    \end{align*}
\end{frame}

\begin{frame}\frametitle{Full Supply Model}
    Putting these pieces together yields an expression for prices in market $t$ as a function of data
    $$\ln(p_{jt} - b_{jt}(x_t, p_t, \delta_t; \theta_2)) = w_{jt} \gamma + \omega_{jt}$$ \\
    which BLP estimate to construct the marginal cost for every automobile model \\
    \vspace{2ex}
    As with the demand model, there is an endogeneity problem
    \begin{itemize}
        \item The price vector is a function of unobserved cost characteristics, $\omega$, and the markup, $b_t$, is a function of prices, so the markup is correlated with unobserved cost characteristics
        \item The unobserved cost characteristics, $\omega$, are correlated with the utility of unobserved product attributes, $\xi$, which are correlated with price
    \end{itemize}
\end{frame}

\begin{frame}\frametitle{Summary of Theoretical Models}
    BLP develop models of demand and supply
    \begin{itemize}
        \item Random coefficients model of demand for differentiated products
        \item Nash-Bertrand pricing model with oligopolistic competition
    \end{itemize}
    \vspace{2ex}
    Both of these models suffer from endogeneity
    \begin{itemize}
        \item BLP need instruments to recover unbiased parameter estimates
    \end{itemize}
    \vspace{2ex}
    The endogeneity enters these models nonlinearly
    \begin{itemize}
        \item BLP develop a new estimation method to handle instruments in a nonlinear model
    \end{itemize}
\end{frame}

\begin{frame}\frametitle{Announcements}
    Upcoming
    \begin{itemize}
        \item Final exam is posted, due December 19
        \item Course surveys are open, due December 22
        \begin{itemize}
            \item \href{http://owl.umass.edu/partners/courseEvalSurvey/uma/}{Click here (\texttt{owl.umass.edu/partners/courseEvalSurvey/uma})}
        \end{itemize}
    \end{itemize}
\end{frame}

\end{document}